\documentclass[a4paper,12pt]{article}
\usepackage{graphicx}
\usepackage{amsmath, amssymb}
\usepackage{hyperref}
\usepackage{geometry}
\geometry{left=1in, right=1in, top=1in, bottom=1in}
\usepackage{caption}
\usepackage{subcaption}
\usepackage{booktabs}
\usepackage{enumitem}
\title{\textbf{Margin Calculation and Risk Analysis for Financial Derivatives}}
\author{Vishal Shisodia}
\date{March 6, 2025}

\begin{document}

\maketitle

This project aims to develop a Python-based tool that facilitates margin requirement calculations and assesses risk exposure for financial derivatives traded on an exchange. The tool automates initial margin computation, dynamically tracks variation margin, and provides essential risk metrics such as Value at Risk (VaR). By streamlining the margining process, this project not only enhances understanding but also offers insights into financial risk management. Furthermore, it aspires to serve as a foundational step towards a more sophisticated risk analysis tool, recognizing that there remains significant scope for refinement and expansion.


\section{Introduction}
Margining plays a crucial role in financial markets by ensuring that traders maintain adequate collateral to cover potential losses. Given my experience as a working student at X(an energy company) in the Exchange, Clearing, and Marketing Operations department, I have developed a deeper understanding of margin requirements and risk assessment. This exposure has inspired me to work on a computational tool that simplifies margin calculations and provides insights into risk exposure. The project aims to bridge the gap between theoretical finance concepts and practical implementation through Python-based automation.

\section{Methodology}
The methodology for this project includes:
\begin{itemize}
    \item \textbf{Data Collection}: Historical price data of natural gas futures is retrieved using the \texttt{yfinance} (Yahoo Finance) library.
    \item \textbf{Initial Margin Calculation}: Based on historical volatility, initial margin is estimated.
    \item \textbf{Variation Margin}: Daily profit/loss is tracked, and variation margin is updated accordingly.
    \item \textbf{Risk Analysis}: The model incorporates Value at Risk (VaR) to measure potential losses.
    \item \textbf{Machine Learning Model}: A Random Forest Regressor is trained to predict initial margin using key risk factors.
    \item \textbf{Visualization}: Trends in price, margin requirements, and risk metrics are visualized using \texttt{matplotlib}.
\end{itemize}

\section{Implementation}
The implementation follows these key steps:
\begin{enumerate}
    \item Fetch historical price data using \texttt{yfinance}.
    \item Compute rolling volatility to estimate initial margin.
    \item Track daily changes in price to update variation margin.
    \item Train a \textbf{Random Forest Regressor} to predict initial margin.
    \item Evaluate the model using \textbf{Mean Squared Error (MSE)} and \textbf{R-Squared Score (R²)}.
    \item Generate visual representations for analysis.
\end{enumerate}

\section{Results and Analysis}
The model successfully computes initial and variation margin, along with basic risk metrics. Below are key observations:
\begin{itemize}
    \item Higher volatility leads to increased margin requirements.
    \item Variation margin reflects daily profit/loss dynamics.
    \item Value at Risk (VaR) provides a probabilistic estimate of potential losses.
\end{itemize}
\begin{figure}[h]
    \centering
    \includegraphics[width=0.85\textwidth]{image.png}
    \caption{Value at Risk (VaR) Analysis: Comparing Historical VaR and Parametric VaR at a 95\% confidence level.}
    \label{fig:var_analysis}
\end{figure}

\subsection{Model Performance}
To evaluate the predictive performance of our machine learning model, we computed the following metrics:
\begin{itemize}
    \item \textbf{Mean Squared Error (MSE)}: 0.00078 (Lower values indicate better predictions)
    \item \textbf{R-Squared Score (R²)}: 0.982 (The model explains 98.2\% of variance in Initial Margin)
\end{itemize}
These results suggest that the model has a high predictive capability, accurately estimating initial margin based on key risk factors.

\subsection{Feature Importance Analysis}
To understand which factors most influence margin requirements, we analyzed feature importance from the trained Random Forest model:
\begin{table}[h]
    \centering
    \begin{tabular}{lc}
    \toprule
    \textbf{Feature} & \textbf{Importance} \\
    \midrule
    Parametric VaR & 0.943214 \\
    Volatility & 0.045167 \\
    Historical VaR & 0.005312 \\
    Volatility 5d Avg & 0.004095 \\
    Var Margin 5d Avg & 0.002749 \\
    Var Margin Change & 0.002463 \\
    \bottomrule
    \end{tabular}
    \caption{Feature Importance in Initial Margin Prediction}
    \label{tab:feature_importance}
\end{table}

The results indicate that \textbf{Parametric VaR} is the most significant factor influencing initial margin, followed by \textbf{Volatility}. Historical VaR and short-term variations in margin have minor contributions.

\section{Model Limitations}
While this model offers a structured approach to margin calculation and risk analysis, it is not entirely accurate as it does not incorporate several broad market factors. Some key limitations include:
\begin{itemize}
    \item It does not implement advanced exchange-level margin methodologies such as SPAN.
    \item The model considers individual asset risk without integrating portfolio-level diversification effects.
    \item Real-time market conditions and news events, which significantly impact margin requirements, are not factored in.
    \item Extreme market shocks and stress testing scenarios are beyond the scope of this model.
\end{itemize}
Despite these limitations, this project serves as a fundamental framework that can be further developed into a more comprehensive risk management tool.

\section{Future Improvements}
To enhance the model, the following improvements can be made:
\begin{itemize}
    \item Implement SPAN-based margining methodologies.
    \item Incorporate real-time data and dynamic updates.
    \item Extend the model to support multi-asset portfolios.
    \item Introduce advanced risk metrics like Expected Shortfall (Conditional VaR).
\end{itemize}

\section{Conclusion}
This project provides a foundational tool for understanding margin requirements and risk exposure in financial derivatives trading. While it simplifies complex financial concepts, it serves as an educational resource with potential for future development. The model is an initial step towards building a more sophisticated and comprehensive risk management tool.

\section{References}
\begin{itemize}
    \item CME Group. ``SPAN Methodology.'' Available at: \url{https://www.cmegroup.com}
    \item Hull, J. ``Options, Futures, and Other Derivatives.'' Pearson Education.
    \item Yahoo Finance API Documentation. Available at: \url{https://www.yfinance.com}
\end{itemize}

\end{document}
